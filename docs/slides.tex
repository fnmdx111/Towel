\documentclass[10pt, compress, handout]{beamer}

\usetheme[usetitleprogressbar]{metropolis}

\usepackage{mdframed}

\usepackage{booktabs}
\usepackage[scale=2]{ccicons}
\usepackage{minted}

\usepackage{alltt}

\newcommand{\sst}[1]{\textcolor{teal}{\textbf{#1}}}
\newcommand{\ltt}[1]{\textcolor{violet}{\textbf{#1}}}
\newcommand{\nmt}[1]{\textcolor{orange}{\textbf{#1}}}

\newcommand{\msst}[1]{\textcolor{teal}{\texttt{#1}}}
\newcommand{\mltt}[1]{\textcolor{violet}{\texttt{#1}}}
\newcommand{\mnmt}[1]{\textcolor{orange}{\texttt{#1}}}

\newcommand{\ssti}[1]{#1}
\newcommand{\ltti}[1]{#1}
\newcommand{\nmti}[1]{#1}

\usemintedstyle{trac}

\title{The Towel Programming Language \includegraphics[scale=0.15]{images.jpg}}
\subtitle{W4115 PLT, Fall 2015}
\author{Zihang Chen (zc2324) Baochan Zheng (bc2269) Guanlin Chen (gc2666)}
\institute{Columbia University}

\begin{document}

\maketitle

\begin{frame}[fragile]
\frametitle{Introduction}
What is Towel anyway?

\pause
It is ...
\begin{itemize}[<+->]
\item $\lambda$ \onslide<+->{: tail recursion, function as first-class citizen, etc.}
\item Stack-based and postfix-syntaxed
\item Dynamically strong-typed
\item General-purpose
  \onslide<+->{\begin{alltt}
    \ltt{42} \nmt{!println}
\end{alltt}}
\end{itemize}
\end{frame}

\begin{frame}[fragile]
  \frametitle{Introduction}
  What does it look like?

\pause
  \begin{mdframed}
\begin{alltt}
  import 'std' @

  bind Fold-left ,\textbackslash Acc Xs Fun,
    (Xs ?# ift Acc,
       (Acc Xs #hd Fun Xs #tl Fun` Fold-left@))
  also Sum (0 (+` Fold-left /flip))
  then ([1 10 11 20] Sum !println)
\end{alltt}
  \end{mdframed}
\end{frame}

\begin{frame}[fragile]
  \frametitle{How to recognize different parts of the example}

  Let me do some syntax-highlighting here.

\pause
  \begin{alltt}
  \sst{import} 'std' \sst{@}

  \sst{bind} Fold-left \sst{,\textbackslash} Acc Xs Fun\sst{,}
    \sst{(}Xs ?# \sst{ift} Acc\sst{,}
       \sst{(}Acc Xs #hd Fun Xs #tl Fun\sst{`} Fold-left\sst{@))}
  \sst{also} Sum \sst{(}0 \sst{(}+\sst{`} Fold-left /flip\sst{))}
  \sst{then} \sst{(}[1 10 11 20] Sum !println\sst{)}
  \end{alltt}
\vspace{0.5cm}

  \begin{itemize}
  \item \sst{Language Structures}:
\small
\msst{Sequence}, \msst{if} forms, \msst{Function}, \msst{Backquote}, \msst{bind-then} forms, \msst{import} form, \msst{export} form
\normalsize
  \end{itemize}
\vspace{0.3cm}

\end{frame}

\begin{frame}[fragile]
  \frametitle{How to recognize different parts of the example}

  Let me do some syntax-highlighting here.

  \begin{alltt}
  \sst{import} 'std' \sst{@}

  \sst{bind} Fold-left \sst{,\textbackslash} Acc Xs Fun\sst{,}
    \sst{(}Xs ?# \sst{ift} Acc\sst{,}
       \sst{(}Acc Xs #hd Fun Xs #tl Fun\sst{`} Fold-left\textcolor{red}{@}\sst{))}
  \sst{also} Sum \sst{(}0 \sst{(}+\sst{`} Fold-left /flip\sst{))}
  \sst{then} \sst{(}[1 10 11 20] Sum !println\sst{)}
  \end{alltt}

\vspace{-0.16cm}
\textcolor{red}{Tail recursive function call}

  \begin{itemize}
  \item \sst{Language Structures}:
\small
\msst{Sequence}, \msst{if} forms, \msst{Function}, \msst{Backquote}, \msst{bind-then} forms, \msst{import} form, \msst{export} form
\normalsize
  \end{itemize}
\vspace{0.3cm}

\end{frame}

\begin{frame}[fragile]
  \frametitle{How to recognize different parts of the example}

  Let me do some syntax-highlighting here.

  \begin{alltt}
  \sst{import} 'std' \sst{@}

  \sst{bind} Fold-left \sst{,\textbackslash} Acc Xs Fun\sst{,}
    \sst{(}Xs ?# \sst{ift} Acc\sst{,}
       \sst{(}Acc Xs #hd Fun Xs #tl Fun\sst{`} Fold-left\sst{@))}
  \sst{also} Sum \sst{(}0 \sst{(}\textcolor{red}{+` Fold-left} /flip\sst{))}
  \sst{then} \sst{(}[1 10 11 20] Sum !println\sst{)}
  \end{alltt}

\vspace{-0.16cm}
\textcolor{red}{Partial function application}

  \begin{itemize}
  \item \sst{Language Structures}:
\small
\msst{Sequence}, \msst{if} forms, \msst{Function}, \msst{Backquote}, \msst{bind-then} forms, \msst{import} form, \msst{export} form
\normalsize
  \end{itemize}
\vspace{0.3cm}

\end{frame}

\begin{frame}[fragile]
  \frametitle{How to recognize different parts of the example}

  Let me do some syntax-highlighting here.
  \begin{alltt}
  \ssti{import} \ltt{'std'} \ssti{@}

  \ssti{bind} Fold-left \ssti{,\textbackslash} Acc Xs Fun\ssti{,}
    \ssti{(}Xs ?# \ssti{ift} Acc\ssti{,}
       \ssti{(}Acc Xs #hd Fun Xs #tl Fun\ssti{`} Fold-left\ssti{@))}
  \ssti{also} Sum \ssti{(}\ltt{0} \ssti{(}+\ssti{`} Fold-left /flip\ssti{))}
  \ssti{then} \ssti{(}\ltt{[1 10 11 20]} Sum !println\ssti{)}
  \end{alltt}
\vspace{0.5cm}

  \begin{itemize}
  \item \ltt{Literals}: \small literals for \mltt{atoms}, \mltt{numbers}, \mltt{strings}, \mltt{lists}, \mltt{tuples} are supported
\normalsize
  \end{itemize}
\vspace{0.3cm}
\end{frame}

\begin{frame}[fragile]
  \frametitle{How to recognize different parts of the example}

  Let me do some syntax-highlighting here.
  \begin{alltt}
  \ssti{import} \ltti{'std'} \ssti{@}

  \ssti{bind} \nmt{Fold-left} \ssti{,\textbackslash} \nmt{Acc} \nmt{Xs} \nmt{Fun}\ssti{,}
    \ssti{(}\nmt{Xs} \nmt{?#} \ssti{ift} \nmt{Acc}\ssti{,}
       \ssti{(}\nmt{Acc} \nmt{Xs} \nmt{#hd} \nmt{Fun} \nmt{Xs} \nmt{#tl} \nmt{Fun}\ssti{`} \nmt{Fold-left}\ssti{@))}
  \ssti{also} \nmt{Sum} \ssti{(}\ltti{0} \ssti{(}\nmt{+}\ssti{`} \nmt{Fold-left} \nmt{/flip}\ssti{))}
  \ssti{then} \ssti{(}\ltti{[1 10 11 20]} \nmt{Sum} \nmt{!println}\ssti{)}
  \end{alltt}
\vspace{0.5cm}

  \begin{itemize}
  \item \nmt{Names}: \small extensive characters supported, flexible naming \\ ~
\normalsize
  \end{itemize}
\vspace{0.3cm}
\end{frame}

\begin{frame}[fragile]
  \frametitle{How to recognize different parts of the example}

  Let me do some syntax-highlighting here.
  \begin{alltt}
  \sst{import} \ltt{'std'} \sst{@}

  \sst{bind} \nmt{Fold-left} \sst{,\textbackslash} \nmt{Acc} \nmt{Xs} \nmt{Fun}\sst{,}
    \sst{(}\nmt{Xs} \nmt{?#} \sst{ift} \nmt{Acc}\sst{,}
       \sst{(}\nmt{Acc} \nmt{Xs} \nmt{#hd} \nmt{Fun} \nmt{Xs} \nmt{#tl} \nmt{Fun}\sst{`} \nmt{Fold-left}\sst{@))}
  \sst{also} \nmt{Sum} \sst{(}\ltt{0} \sst{(}\nmt{+}\sst{`} \nmt{Fold-left} \nmt{/flip}\sst{))}
  \sst{then} \sst{(}\ltt{[1 10 11 20]} \nmt{Sum} \nmt{!println}\sst{)}
  \end{alltt}

  \begin{itemize}
  \item \sst{Language Structures}
  \item \ltt{Literals}
  \item \nmt{Names}
  \end{itemize}

The above three are what we call \textbf{word}s in Towel. A program in Towel is essentially a \textbf{sentence} of words.
\end{frame}


\newcommand{\tpt}[1]{\textcolor{magenta}{\texttt{#1}}}

\begin{frame}
  \frametitle{Types in Towel}

Towel supports the following type:
\pause
  \begin{itemize}
  \item<2-> \tpt{Int} $\rightarrow$ Big integer
  \item<2-> \tpt{FixedInt} $\rightarrow$ Signed 64-bit integer
  \item<2-> \tpt{UFixedInt} $\rightarrow$ Unsigned 64-bit integer
  \item<2-> \tpt{Float} $\rightarrow$ IEEE754 floating point
  \item<3-> \tpt{Atom}

    $\rightarrow$ A constant with a name (see also Erlang atoms)
  \item<4-> \tpt{String}

    $\rightarrow$ String (one of the Enumerable types)
  \item<4-> \tpt{List}

    $\rightarrow$ Linked list (one of the Enumerable types)
  \item<4-> \tpt{Tuple}

    $\rightarrow$ Fixed, random accessible enumerable data type
  \item<5-> \tpt{Function} 

    $\rightarrow$ Passing around a piece of code
  \end{itemize}
\end{frame}

\begin{frame}[fragile]
  \frametitle{The Towel Standard Library, a.k.a. Towelibs}

  In module \mnmt{Std}, you will find ...
  \pause
  \begin{itemize}[<+->]
  \item Arithmetic Functions: \mnmt{+}, \mnmt{-}, etc. So \textbf{no operators}.
  \item Conversion and Reflection Functions: \mnmt{\textasciitilde int}, \mnmt{\textasciitilde str}, etc.
  \item Routines: functions with side(or stack)-effects, e.g. \mnmt{!println}, \mnmt{!!pop}, \mnmt{!!dup}, etc.
  \item Functions that work with enumerables: \mnmt{\#hd}, \mnmt{\#tl}, \mnmt{\#cons}, etc.
  \item The Fun Functions: \mnmt{/foldl}, \mnmt{/map}, \mnmt{/filter}, etc.
  \item Variadic Functions: a pacman that eats arguments until the stack is empty. See manual for more detail.
  \end{itemize}
\end{frame}

\begin{frame}[fragile]
  \frametitle{The Towel Compiler, codename \texttt{weave}}

  How \texttt{weave} compiles a piece of \textit{towel}: it ...
  \begin{enumerate}[<+->]
  \item \textcolor{black}{Source} $\rightarrow$ \textcolor{gray}{Tokens}

    tokenizes the source code using a scanner
  \item \textcolor{gray}{Tokens} $\rightarrow$ \textcolor{orange}{AST}

    parses the tokens with a parser
  \item \textcolor{orange}{AST} $\rightarrow$ \textcolor{olive}{IR AST}

    traverses and transforms AST to IR AST (along with some scope analysis that will detect unbound names)

  \item \textcolor{olive}{IR AST} $\rightarrow$ \textcolor{blue}{Bytecode}

    compiles IR AST into bytecode representation
  \end{enumerate}

  \onslide<+->{\textcolor{blue}{Bytecode} is runnable via the Towel Virtual Machine!}
\end{frame}

\begin{frame}[fragile]
  \frametitle{The Towel Virtual Machine}

  The Towel Virtual Machine is a piece of software that ...
  \begin{enumerate}[<+->]
  \item \textcolor{blue}{Bytecode} $\rightarrow$ \textcolor{olive}{IR AST}

     decompiles bytecode to IR AST

   \item \textcolor{olive}{IR AST} $\rightarrow$ \textcolor{cyan}{42}

     interprets the IR AST (essentially an array of instructions) one by one so you can get the answer
  \end{enumerate}

\onslide<+->{You can use the \textcolor{blue}{Extension} mechanism to call OCaml functions from within the Towel Virtual Machine! See manual for more detail.}
\end{frame}

\begin{frame}
  \frametitle{The future of Towel}

  \begin{itemize}
  \item A native compiler that compiles IR to C code.

  \item Better error messages, both for the compiler and the virtual machine.

  \item Better debugging facilities: need to make use the dynamicness feature of Towel.

  \item Enrich the standard library so that it's batteries-included and \textbf{general-purpose}.

  \item Statically typed Towel!

    A stack-based language is very dynamic due to its unclearness of the data (i.e. type) flow. A static-typed Towel could be made by analyzing each function's stack-effect.
  \end{itemize}
\end{frame}

\begin{frame}[fragile]
  \frametitle{And now for something completely different...}

  \huge
  The DEMO

  \begin{figure}[H]
    \includegraphics[scale=0.5]{sillywalk.jpeg}
  \end{figure}
  \normalsize
  \begin{itemize}
  \item Partial function application
  \item Tail calls
  \item Standard library
  \item Extensions to the Towel Virtual Machine
  \item The test suite
  \item Anything you would like to ask
  \end{itemize}
\end{frame}

\end{document}

